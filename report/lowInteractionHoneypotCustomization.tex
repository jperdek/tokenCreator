\documentclass[conference, 11pt,slovak,a4paper,twoside]{IEEEtran}
\IEEEoverridecommandlockouts
% The preceding line is only needed to identify funding in the first footnote. If that is unneeded, please comment it out.
\usepackage{cite}
\usepackage{amsmath,amssymb,amsfonts}
\usepackage{algorithmic}
\usepackage{graphicx}
\usepackage{xcolor}
\usepackage{textcomp}
\usepackage{float}
\usepackage{listings}
\usepackage{url}
\lstset{
  basicstyle=\ttfamily,
  columns=fullflexible,
  frame=single,
  breaklines=true
}

\def\BibTeX{{\rm B\kern-.05em{\sc i\kern-.025em b}\kern-.08em
    T\kern-.1667em\lower.7ex\hbox{E}\kern-.125emX}}
\begin{document}

\title{Detekcia nepovoleného prístupu kustomizáciou nízko interaktívnych honeypotov}

\author{\IEEEauthorblockN{Jakub Perdek}
\textit{Slovenská technická univerzita v Bratislave}\\
Bratislava, Slovensko \\
perdek.jakub@gmail.com}

\maketitle

\begin{abstract}
Prevencia pri zabezpečení systému obvykle nemusí byť dostatočná. Z toho dôvodu je potrebné detegovať aktivity spojené hlavne s neoprávneným prístupom do systému, súborom alebo do intranetu. Ich skoré odhalenie môže pomôcť prijať vhodné opatrenia a zabrániť reálnemu útoku na systém. Vhodným nástrojom sú práve honeypoty. Tie naviac môžu slúžiť aj na spomalenie neoprávnených aktivít alebo na zmiatnutie útočníka. Často sú ale ľahko rozpoznateľné od reálnych aktív, a preto je ich kustomizácia nevyhnutná. Lákanie možno realizovať podľa viacerých stratégií, a podľa toho aj prispôsobiť generovanie kustomizovateľných tokenov. Predstavujeme preto automatické riešenie pre tvorbu nízko interaktívnych honeypotov založené na čo najväčšej infiltrácii logiky pre sledovanie rozličných foriem manipulovania s nimi v rámci konkrétnej biznis logiky. Snahou je poskytnúť informácie pri záujeme o konkrétny obsah. Analyzované sú preto rôzne spôsoby aplikovania uvedených mechanizmov v rámci webových dokumentov. Možnosti pre zamedzenie ich zneužitia a obmedzenie manipulácie s nimi by v rámci tohto druhu honeypotov mali byť ľahšie dosiahnuteľné a prispôsobené pre konkrétny prípad použitia.   
\end{abstract}

\begin{IEEEkeywords}
nízko interaktívne honeypoty, detekcia nepovoleného vstupu, kustomizácia honeypotov, webové honeytokeny
\end{IEEEkeywords}

\section{Úvod}
Honeypoty sú bezpečnostným zdrojom, ktorý generuje upozornenie pri zachytení nadviazania interakcie s ním \cite{sanders_intrusion_2020}. Napríklad v podobe prieskumu, útoku alebo pri kompromitácii. Často lákajú infiltrovaný subjekt, a pri ich dobrom maskovaní a umiestnení môžu pomôcť odhaliť neoprávnený prístup, odhalené prístupové údaje, dokonca spomaliť aktivity útočníka vrátane odhalenie nultého útoku a rovnako aj zistiť niektoré ním aplikované postupy pri útoku. Bežní používatelia by nemali s honeypotom vôbec interagovať, ale ani vedieť  o tejto funkcionalite. Prihlásenie sa do takéhoto systému alebo otvorenie a manipulácia s dokumentom, respektíve honey tokenom \cite{ng_honeypot_2018} je preto automaticky podozrivá a malo by pomocou oznámení byť na ňu upozornené.



\section{Honeypoty pre detekciu neoprávneného prístupu}

\section{Benefity nízko interaktívnych honeypotov pre detekciu neoprávneného prístupu}

Medzi úrovňou interakcie honeypotov je často potrebné robiť trade-off. Napríklad nízko úrovňové honeypoty je jednoduché nainštalovať, ľahko emulujú niektoré služby. Riziko je tak nízke ale rovnako je limitovaná aj poskytnutá informácia o aktivite útočníka \cite{spitzner_honeypots_2002}. Často nevyžadujú zmeny v existujúcej topológii siete alebo na zariadeniach \cite{scada_2014}. Nízko interaktívne honeypoty sú väčšinou produkčné honeypoty určené na ochranu organizácie \cite{spitzner_honeypots_2002}.


\section{Návrh nízko interaktívneho honey tokenu z webových dokumentov}

Nízko interaktívne honeypoty pre detekciu neoprávneného prístupu by mali zahŕňať mechanizmy pre generovanie informácie o príslušnej udalosti. Tie by mali byť dostatočne maskované, tak aby sa útočník o svojom odhalení najlepšie ani nedozvedel. V svojej práci sme sa zamerali na webové dokumenty. 


\subsection{Mechanizmy pre logovanie aktivity s webovým tokenom}

Analyzovali sme rôzne spôsoby vloženia spomenutého mechanizmu do webového dokumentu. Zamerali sme sa na prípady, keď sa manipuluje s webovým dokumentom neposkytovaným priamo v rámci nejakého spusteného servera. Výsledný honeytoken v podobe konkrétneho dokumentu bude poskytovať obmedzené možnosti interakcie s ním, a tým sa zabráni aj jeho zneužitiu. Identifikovali sme nasledovné prípady umožnenie sledovania interakcie s uvedeným dokumentom:

\begin{enumerate}
	\item Použitie nástroja sledujúceho zmeny v rámci súborového systému, prípadne iného systému v rámci ktorého sa uvedené tokeny manažujú, pre vybrané súbory. Získanou informáciou potom je len informácia o záujme o tieto súbory. Detekovanie takejto zmeny môže byť aj jednoduchšie detegovať. 
	
	\item Vloženie, respektíve nahradenie alebo upravenie existujúceho iframe elementu novým. Infiltrovanej osobe sa tým priamo poskytne hľadaný obsah aj s odoslaním informácií pri dopytoch po zdrojových súboroch nového obsahu. Zároveň možno odpozorovať o aký obsah má osoba záujem a príslušne tento obsah kustomizovať. Aj napriek tomu, ža sa iframy už neodporúča používať stále sa vyskytujú hlavne pri službách poskytujúcich letáky v internetových obchodoch.

	\item Vytvorenie a odoslanie informujúcej správy, respektíve logu na server. Správu by malo byť potrebné maskovať. Oproti predchádajúcemu spôsobu môže byť takáto správa odchytená a jej odoslanie na server útočníkom znemožnené, keďže sama o sebe spravidla neposkytuje pre útočníka relevantný obsah. Mala by preto byť prepojená s vyžiadaním konkrétnych kľúčov pre sprístupnenie tohto obsahu. 
\end{enumerate}
 
Obmedzenie na menované spôsoby je hlavne z dôvodu ochrán pred internetovými hrozbami,v rámci ktorých nie je povolené zapisovať na disk pomocou prehliadaču štandardným spôsobom. Rovnako ukladanie je obmedzené na použitie cookies, ktoré si používateľ spravidla môže prezerať. Pri aplikácii preto zostáva informovať s použitím naviazania na zdroje, ktoré sú ku dokumentu priradené.
  

\subsection{Nástroje pre maskovanie logiky honeypotu a ich význam pre kustomizáciu}

Pri tvorbe funkcionality umožňujúcej detekovať manipuláciu s týmito tokenmi a ich ďalšiu kustomizáciu sme použili nástroje pre:

\begin{enumerate}
	\item Vytvorenie kópie danej webovej lokality, prípadne iba jej časti. 

	\item Manipulovanie so štruktúrou webového dokumentu a automatickú zmenu informácií v rámci nej.
	
	\item Minifikáciu a zamlženie súborov pre kaskádové štýly, skripty pre javascript a hlavne HTML dokumenty. 

	\item Využitie proxy, ktoré umožňuje získať a pozmeniť časť pravého obsahu pre nalákanie útočníka.
	
	\item Zahashovanie a zneprístupnenie obsahu až do momentu priameho vykonania skriptov na tránke. Inak by útočník mohol získať želaný obsah iba otvorením konkrétneho súboru.
\end{enumerate}

Vytvorenie kópie webovej lokality sme realizovali pomocou balíčkov prístupných v jazyku python. Zistili sme ale ich obmedzenú funkčnosť pri reálnom použití. V prípade balíčku pywebcopy~\footnote{\url{https://pypi.org/project/pywebcopy/}} to bolo časté zamrznutie skriptu kvôli zlúčeniu vlákien. Použili sme preto najnovšiu verziu priamo z githubu~\footnote{\url{https://github.com/rajatomar788/pywebcopy}} a zablokovali použitie multivláknového spracovania ale vyskytol sa nový problém s absolútnymi adresami skopírovaného dokumentu. Kvôli nim sa napríklad nezobrazia pôvodné obrázky na stránke. Ďalším nástrojom bolo vloženie podpory pre HTTrack~\footnote{\url{https://www.httrack.com}}, ktorý je efektívnejší a nemal komplikácie ako predchádzajúci balík. Jeho použitie je ale závislé na operačnom systéme a v súboroch necháva informácie o použití tohto nástroja.

Príklad použitia nástroja HTTrack pre získanie súborov do hĺbky 1, bez externých závislostí, a ich uloženie do priečinku ./downloads:

\begin{lstlisting}
"C:\\Program Files\\WinHTTrack\\httrack.exe" https://www.trony.it/online/store-locator -v -r1 -\%e0 -O ./download
\end{lstlisting}


Pokiaľ je možné tak manipulujeme so štruktúru webového dokumentu dynamicky s použitím knižnice Beautifulsoup. Typickou použiadavkou je upravenie konkrétnych iframe elementov a nastavenie ich atribútov. Niekedy je potrebné funkcionalitu využiť pre prepojenie aj upravených alebo vytvorených skriptov alebo kaskádových štýlov s webovým dokumentom. Pokiaľ načítanie tejto štruktúry nie je možné malo by byť zabezpečené pridanie uvedených častí v textovej podobe.

Do riešenia sme pridali niekoľko nástrojov pre zamlženie a minfikáciu súborov. Rôzne nástroje poskytovali funkcionalitu pre rozdielne typy súborov. Pre minifikáciu HTML sme použili pythonovskú knižnicu htmlmin~\footnote{\url{https://pypi.org/project/htmlmin}} s nastavením pre vymazanie prázdneho miesta a odstránenie komentárov. Pre minimalizáciu javascriptu sme použili Closure Compiler~\footnote{\url{https://github.com/google/closure-compiler}} od Googlu. V rámci nastavení sme použili compilation-level na SIMPLE\_OPTIMIZATIONS, kvôli tomu, že pri pokročilejšom nastavení dochádzalo k odstránenie všetkého kódu pokiaľ nebol priamo exportovaný. Redundantné funkcie určené na zneprehľadnenie kódu by tak boli odstránené. Následne sme pridali parameter pre spracovanie aj javascriptovských súborov v striktnom móde. Minimalizáciu kaskádových štýlov sme zabezpečili pomocou externého nástroja Yui Compressor~\footnote{\url{https://github.com/yui/yuicompressor}} volaného podobne z príkazového riadku. Jednotlivé nástroje sme aplikovali na príslušné súbory v konkrétnej vybranej stromovej štruktúre. V praxi takýmito súbormi sú súbory klonovanej webvej lokality so zapracovanou sledovacou logikou.

Ďalšou voliteľnú časť tvorí použitie proxy pre poskytnutie časti relevantného obsahu s dodatočnou možnosťou úpravy. Funkcionalitu sme sa rozhodli pridať kvôli uľahčeniu procesu kopírovania časti namiesto celej webovej lokality, keďže v praxi táto činnosť vyžadovala kopírovanie obsahu aj z externých stránok. Reštriktívny je CORS hlavne pri zabezpečenejších aplikáciám. Namiesto proxy je preto výhodnejšie poskytovať celý fiktívny obsah pre honeypot priamo. Pri využití proxy možno reálne dáta z existujúcich služieb upravovať za behu, ale rastie riziko odhalenia takéhoto honeypotu.

Podstatnými sú nástroje pre maskovanie obsahu. Dáta je možné uložiť do viacerých častí, ktoré sa v priebehu vykonania zložia. To núti používateľa aby súbor reálne otvoril v prehliadači, nechal vykonať kód a tým zapríčinil aj vykonanie skrytej sledovacej logiky. Opäť by sprístupnenie obsahu malo spočívať na komunikácii so serverom, inak sledovacia logika neoznámi manipuláciu s tokenom. Rovnako je možné použiť base64 hash, ktorým ukryjeme obsah a hlavne sledovaciu logiku pred vyhľadaním zvonka. Alternatívou je prevod do hexadecimálneho formátu alebo použitie knižnice pre obalenie dát. Existujú rôzne formáty, používaným môže byť gzip. 


\section{Automatizácia kustomizácie honeytokenu}

V rámci automatizácie a kustomizácie honey tokenu riešime dve stratégie. V rámci prvej používateľ špecifikuje počet výrazných honeytokenov a doménu z ktorej majú byť vytvorené. V rámci druhej stratégie automatizácia a kustomizácia nebude zvýrazňovať určité charakteristiky cieľového súboru a jeho obsahu ale vytvorí dostatočný počet takýchto podobných tokenov.

V rámci samotnej automatizácie sa vykonajú kroky v nasledovnej postupnosti:
\begin{enumerate}
	\item Klonovanie webových dokumentov z danej domény. Zváži sa tu aj klonovanie nepovolených stránok v rámci súboru robots.txt a hĺbka tohto klonovania.

	\item Pre každý generovaný prvok sa rovnako vykoná nasledovný postup. Vloží sa element so sledovacím iframe, pri ktorom server loguje požiadavky pre dopytovanie sa po jeho obsahu alebo sa nahradí existujúci iframe aj spolu s presunutím poskytovania tohto obsahu na server. V prípade nepoužitia iframu sa vygeneruje a vloží kód odosielajúci informácie pri vyžiadaní konkrétneho obsahu.
	
	\item Klonovanie webovej lokality na adrese, na ktorú smeruje iframe a tvorba API pre spracovanie funkcionality poskytovanej konkrétnym iframom. Obsah v rámci biznis domény uložený v rámci konkrétnych súborov by mal byť ďalej kustomizovaný. V prípade nepoužitia iframu sa vygeneruje kostra servera poskytujúceho kľúče pre sprístupnenie obsahu z klienta a odmaskovanie správy pre jej zalogovanie.
	
	\item Ďalšie hashovanie a zneprístupňovanie obsahu v rámci súborov. Zabezpečenie funkčnosti pri vykreslení stránky a aplikovaní kľúčov zo serveru.
	
	\item Prípadné pridanie redundantného kódu alebo spájanie a vykonanie zamlženej verzie.
	
	\item Rovnako pre každý generovaný honeytoken prebehne minifikácia a prípadné zamlženie skriptov, html súborov a asociovaných kaskádových štýlov. 
\end{enumerate}


\section{Návrh a implementácia nástrojov pre utajenie obsahu}

Ukrytie obsahu konkrétnych správ ale aj celých skriptov môže byť v rámci statických súborov nevyhnutné. Zároveň je potrebné aplikovať takýto postup na každú zahrnutú biznis informáciu, ktorá by neskôr mala byť sprístupnená a prezentovaná v rámci konkrétneho honeytokenu. Sprístupnia sa potom už iba konkrétne fiktívne biznis informácie. Útočník by si konkrétnych logov a ich obsahu potom nemal všimnúť. Rovnako by nemal ani neblokovať požiadavky smerujúce na server kvôli jeho potencionálnemu záujmu o biznis obsah. Uvedená funkcionalita by mala zabezpečiť rôzne druhy informácií odosielané z rôznych častí webového dokuemntu a zvýšiť tak možnosti informovania o činnosti útočníka.


\subsection{Utajenie obsahu kompresiou a kódovaním}

Pre budúci generátor honey tokenov sme navrhli funkcionalitu, ktorá bude obsah niekoľko ráz maskovať s použitím náhodne vybraných metód uplatnených aj niekoľko krát. Výsledkom bude maskovaný obsah a kľúčom bude postupnosť identifikátorov metód, aby následne mohli byť použité pre navrátenie textu do čitateľnej podoby. Funkcionalitu sme zostrojili tvorbou triedy podporujúcej dve základné metódy pre maskovanie a odmaskovanie. Tie konkrétne implementácie dediace od nej prekryli vlastnými využívajúcimi svoju funkcionalitu a umožnili tak náhradu za túto základnú triedu, čo dopomohlo k náhodnému výberu konkrétnej metódy. V základne triede sme implementovali nástroje pre náhodný výber metódy, získanie metódy podľa identifikátora, funkcionalitu umožňujúcu maskovať a rovnako odmaskovať konkrétny obsah.

Pri implementácii konkrétnych maskovacích metód sme využili knižnice tretích strán. Okrem základnej funkcionality umožňujúcej vytvoriť z textu hash ako napríklad base64, base32, base16 alebo base85 (knižnica base64~\footnote{\url{https://docs.python.org/3/library/base64.html}}) sme implementovali aj podporu pre kompresné metódy ako je gzip (knižnica gzip~\footnote{\url{https://docs.python.org/3/library/gzip.html}}), deflate (knižnica zlib~\footnote{\url{https://docs.python.org/3/library/zlib.html}}) alebo brotli (knižnica brotli~\footnote{\url{https://pypi.org/project/Brotli}}). Vstupom ako aj výstupom každej z týchto metód je textový reťazec. Problémom kompresných metód je vrátenie výslednej hodnoty v binárnej podobe. Bolo preto zaručiť konverziu do textovej podoby. To ale v rámci kódovania utf-8 nebolo kvôli rôznym nevyhovujúcim znakom možné. Pri vynechaní chýb nebolo spätné odmaskovanie úplne úspešné a uplatnenie viacerých metód ani možné. Rovnako pretypovanie spôsobovalo chyby pri integrácii spomenutých kompresných metód. Nakoniec sa nám problém podrilo vyriešiť nástrojom pre quotovanie url (knižnica urlib~\footnote{\url{https://docs.python.org/3/library/urllib.html}}), ktorý to dokázal urobiť aj z binárnej podoby (metóda pre prevod z binárnej podoby na text má podobu urllib.parse.quote\_from\_bytes(compressed\_bytes) a metóda pre spätný prevod má podobu urllib.parse.unquote\_to\_bytes(encoded\_text)). Ako ďalšie metódy pre maskovanie textu sme použili funkcionalitu pre quotovanie url (urllib.parse.quote\_plus(pure\_text)) a jeho spätné dekódovanie (urllib.parse.unquote\_plus(encoded\_text)).

Nakoniec sme implementovali spomenuté metódy pre maskovanie a odmaskovanie. Metóda pre maskovanie získa na vstupe zvolený text, pole maskovacích metód a počet uplatnených cyklov. Následne náhodne vyberie z tohto poľa maskovaciu metódu a uplatní ju na zvolený text. Zaznamená sa aj druh tejto metódy podľa identifikátora tejto metódy. Postup sa opakuje zvolený počet krát vždy s predtým maskovaným textom. Výsledkom je maskovaný text a identifikátor všetkých použitých maskovacích metód ako kľúč. Ten sa môže umiestniť na serveri a zabezpečiť tak utajenie obsahu, alebo zaručiť odmaskovanie už na klientovi s nevyhnutnou potrebou maskovania tohto kľúču. V rámci uvedenej funkcionality je predpoklad odmaskovania na klientovi dôležitý, inak by postačovala aj iná forma šifrovania, pri ktorej by server bol schopný sprístupniť konkrétne správy, respektíve informácie.

Metóda na odmaskovanie na vstupe požaduje maskovaný text a identifikátor všetkých použitých metód pre maskovanie. V rámci tejto metódy sa vykoná reverzná aplikácia uvedených metód v opačnom poradí ako boli použité pri maskovaní. Výsledkom by mal byť odmaskovaný text v podobe v akej bol pred jeho utajením.


\subsection{Zamlženie kódu jeho rozdelením a vykonaním v rámci druhého kódu}

V praxi okrem štandardnej minifikácie kódu zahŕňajúcej náhradu premenných a odstránenia prázdneho miesta môže byť užitočné aj rozdelenie kódu na viaceré časti. Ak sa pred rozdelením aplikuje maskovanie obsahu z predchádzajúcej časti dostávame tak dômyselne utajený obsah. Týmto obsahom by mala byť najčastejšie správa o postupe útočníka ale môže ním byť aj zdrojový kód samotný. V prípade takéhoto spracovania zdrojového kódu samotného ako reťazca je potrebné vykonanie metódy eval. V rámci nej sa zdrojový kód zapísaný ako textový reťazec vykoná. Použitie tejto metódy je stále veľkým bezpečnostným riskom, a preto by sa malo takému použitiu vyhnúť. Ideálne kód predĺžiť rozdelením jednotlivých príkazov, lepším prepletením kódu s bizis logikou alebo pridaním nepotrebných a redundantných častí.

V rámci našej implementácie sme umožnili rozdeliť textový reťazec na viaceré časti s použitím premenných a polí. Do polí sú tieto časti pridávané priamo alebo vo forme premenných, ku ktorým sú predtým priradené. Následne je predtým rozdelený reťazec znovu zložený. Jednotlivé príkazy sú generované zaradom a ukladané do poľa. Náhodne sa pritom aplikujú implementované metódy pre generovanie príkazov ako napríklad tvorba premennej vo forme textového poľa alebo textového reťazca a zlúčenie premenných. Dodatočne je možné pridávať medzi tieto príkazy aj iné redundantné príkazy alebo príkazy prislúchajúce ku konkrétnej biznis logike. Funkcionalitu sme testovali s aplikovaním maskovania obsahu a testovali s vykonaním javascriptu pomocou knižnice js2py~\footnote{\url{https://pypi.org/project/Js2Py/}}. Výstupné texty boli zhodné. Identifikovali sme aj problém s použitím deklarácií s použitím let, pri ktorých bolo potrebné konkrétny kód obaliť do funkcie inak mohlo dochádzať ku kolízii premenných. Použitie var namiesto let by problém vyriešilo, ale mohlo by spôsobiť aj problémy a malo by byť používané len výnimočne pre globálne deklarácie. 


\section{Vloženie detekčnej logiky}

Detekčná logika je hlavným obsahom honeytokenu. Do dokumentu musí byť vhodným spôsobom pridaná a následne je potrebné zabezpečiť zneprístupnenie informácií a zamedziť jej odhalenie.


\subsection{Vloženie detekčného skriptu}

Podstatným pre využitie a prepojenie predtým realizovaných metód pre klonovanie, minimalizáciu a utajovanie obsahu je samotné vloženie správy alebo skriptu. To musí byť maskované vzhľadom na požiadavku utajene informovať o konkrétnom obsahu. Celý proces realizujeme v niekoľkých krokoch, pričom využívame náhodné generátory pre vygenerovanie rôznych parametrov určujúci finálny výsledok:


\begin{enumerate}
	\item Príprava skriptu pre odosielanie hlásení na server. Dodatočne je možné pripraviť aj zoznam správ, ktoré sa budú posielať.

	\item Príprava webových služieb pre manažovanie logov a samotnej kostry serveru.
	
	\item Klonovanie príslušnej časti webovej lokality.
	
	\item Načítanie skriptu pre detegovanie podozrivej aktivity, prípadne aj samostatných správ. 
	
	\item Zamaskovanie buď celého skriptu, pričom v neskorších fázach bude potrebné použiť funkciu eval pre jeho vykonanie po odmaskovaní. Dôležité je maskovanie jednotlivých správ samostatne kvôli ich utajeniu. Tie budú následne počas vykonávania odosielané. 
	
	\item Vloženie metód pre dešifrovanie konkrétneho obsahu s kódom metódy, ale nie samotných správ s dôrazom na maskovanie identifikátora použitých metód. Implementovanie tejto funkcionality priamo na serveri z dôvodu zistenia obsahu logov a ich následného spracovania.
	
	\item Rozdelenie obsahu do niekoľkých riadkov kódu, prípadne metód. Vhodné je aj zahrnúť tento obsah do kódu pre biznis logiku alebo pridať redundantný kód.
	
	\item V prípade maskovania celého skriptu je potrebné po domaskovaní zabezpečiť volanie funkcie eval pre jeho vykonanie.
	
	\item Vloženie celého kódu do script elementu a pripojenie ho do honeytokenu, respektíve webového dokumentu.
	
	\item Otestovanie spustením servera a otvorením honeytokenu. Odoslané hlásenia by mali byť zaznamenané.
\end{enumerate}


\subsection{Injekcia iframe elementu s konkrétnym obsahom}

Chýbajúci dynamický obsah možno v rámci statického obsahu honeytokenu vložiť ako vnorený dokument pomocou iframe elementu. V konkrétnom dokumente konkrétnej webovej lokality môžme nahradiť už nejaký existujúci alebo vložiť nový. Pri nahradzovaní existujúceho by sme mali nahradiť aj celú poskytovanú biznis logiku. V niektorých prípadoch pre tieto účely možno použiť proxi server. Druhou možnosťou je propojiť iframe element, ktorý sa nemusí ani zobraziť. V tomto prípade vzrastá potencionálna hrozba detegovateľnosti honeypotu. Poskytnutie reálneho obsahu tak môže byť spoľahlivejšie a zaručí to aj zabránenie blokovaniu požiadaviek na server. Aj v tomto prípade uvádzame postup tvorbu takéhoto honeytokenu s dôrazom na náhodné určenie jednotlivých parametrov pri jeho tvorbe:

\begin{enumerate}
	\item Klonovanie príslušnej časti webovej lokality s webovým dokuemntom.

	\item Vyhľadanie iframe elementu a zamenenie src atribútu za vlastnú webovú službu. Ak taký eleemnt neexistuje, alebo je komplikované napodobniť inú lokalitu, potom sa vytvorí nový iframe element s nastavenými rozmermi na 0.
	
	\item Pokiaľ bol nahradzovaný obsah iframe elemntu, potom sa zrealizuje klon webovej lokality na ktorú smeroval jeho pôvodný obsah a vloží sa do zdrojov servera alebo sa prispôsobí server tak aby robil proxy medzi pôvodnou lokalitou. V rámci proxy sa niektoré informácie môzu pozmeniť, aby útočník nedostal reálne biznis dáta.
	
	\item Doplnenie potrebného obsahu pre logovanie a ďalších potrebných služieb (napríklad prípadné proxy) na server.
	
	\item Maskovanie zvyšného obsahu vyššie opísanými technikami pre zmiatnutie útočníka.
	
	\item Otestovanie funkcionality spustením servera a otvorením honeytokenu. Správa o prístupe k danému obsahu by mala byť zaznamenaná.
\end{enumerate}



\section{Zhrnutie a budúca práca}


\bibliographystyle{abbrv} % plain or alpha are fine, too
\bibliography{honeypots}


\end{document}
